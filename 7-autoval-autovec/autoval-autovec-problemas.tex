\documentclass[11pt]{article}

\usepackage{fontspec}
\usepackage{minted}
\usepackage[scale=1]{ccicons}
\usepackage{metalogo}
\usepackage{xcolor,colortbl}
\usepackage{multicol,multirow,booktabs}
\usepackage{graphicx}
\usepackage{bm}
\usepackage{fontawesome}
\usepackage{exsheets}
\usepackage[paper=a4paper, headheight=110pt,showframe=false,
	layoutvoffset=2em,
	bottom=2cm, top=3.5cm]{geometry}
\usepackage[spanish, es-nodecimaldot]{babel}
\usepackage[babel]{microtype}
\usepackage{hyperref}
\usepackage{amsmath}
\usepackage{mismath}
\usepackage{mathrsfs}
\usepackage{gensymb,amssymb}
\setlength{\parindent}{3em}
\setlength{\parskip}{1em}
\usepackage[shortlabels]{enumitem}
\usepackage{subcaption}
\usepackage{wrapfig}
\usepackage[svgnames]{xcolor} % Gestión de colores
%\usepackage{mathspec}
% \usepackage{unicode-math}


% Fonts can be customized here.
\setmainfont[Ligatures=TeX]{Linux Libertine O}
\setmonofont[Scale=0.90,Ligatures=TeX]{Hack Nerd Font Mono}
\usepackage{hyperref}
\hypersetup{
	colorlinks=true, linktocpage=true, pdfstartpage=3, pdfstartview=FitV,%
	breaklinks=true, pageanchor=true,%
	pdfpagemode=UseNone, %
	plainpages=false, bookmarksnumbered, bookmarksopen=true, bookmarksopenlevel=1,%
	hypertexnames=true, pdfhighlight=/O,%nesting=true,%frenchlinks,%
	urlcolor=Maroon, linkcolor=RoyalBlue, citecolor=Blue, %pagecolor=RoyalBlue,%
	pdftitle={},%
	pdfauthor={\textcopyright\ C. Manuel Carlevaro},%
	pdfsubject={},%
	pdfkeywords={},%
	pdfcreator={XeLaTeX}%
}

%% Operadores
\DeclareMathOperator{\sen}{sen}
\DeclareMathOperator{\senc}{senc}
\DeclareMathOperator{\sign}{sign}
\newcommand{\T}[1]{\underline{\bm{#1}}}
\DeclareMathOperator{\Tr}{Tr}
%\NewDocumentCommand{\evalat}{sO{\big}mm}{%
%\IfBooleanTF{#1}
%{\mleft. #3 \mright|_{#4}}
%{#3#2|_{#4}}%
%}


\title{Cálculo avanzado}
\author{Dpto. de Ingenería Mecánica}
\date{Clase 9: Autovalores y autovectores}


\begin{document}
% \maketitle

\begin{center}
\framebox[1.0\textwidth][c]{
\huge{\textsc{Cálculo Avanzado}} 
}
\end{center} 

\begin{center}
\vspace{\baselineskip}
\Large{\textsc{Departamento de Ingenería Mecánica}} \\
\textsc{Facultad Regional La Plata} \\
\textsc{Universidad Tecnológica Nacional}
\end{center}

% \vspace{1em}

\begin{center}
\begin{tabular}{r l}
    \textbf{Práctica:} & 7 \\
 \textbf{Tema:} & Autovalores y autovectores. \\
 \textbf{Profesor Titular:} & Manuel Carlevaro \\
 \textbf{Jefe de Trabajos Prácticos:} & Diego Amiconi \\
\end{tabular}\end{center}

\vspace{1em}
\begin{question}
Compruebe que 
\[ \bm{v} = \begin{bmatrix} 7 \\ 5 \end{bmatrix} \]
es un autovector de 
\[ \bm{A} = 
    \begin{bmatrix}
        53 & -70 \\
        35 & -46
    \end{bmatrix} \]
    ¿Cuál es el autovalor asociado?
\end{question}

\begin{question} % Problemas resueltos Murcia p.2 pg 36

Para la matriz cuadrada $\bm{A}$:
    \[ \bm{A} = \begin{bmatrix} 179 & -99 \\
                255 & -139 \end{bmatrix} \]
el vector $\bm{v} = [3, 5]^{\intercal}$ es un autovector. Compruebe esta afirmación y determine el autovalor asociado.
\end{question}

\begin{question} % Problemas resueltos Murcia p.3 pg 36
    Si $\bm{u}$ y $\bm{v}$ son dos autovectores de la matriz cuadrada $\bm{A}$, ambos asociados al autovalor $\lambda$:
    \begin{enumerate}[a)]
        \item ¿Es $\bm{u} + \bm{v}$ un autovector de $\bm{A}$?. En caso afirmativo, ¿cuál es su autovector?
        \item ¿Es $\bm{v}$ un autovector de la matriz $3 \bm{A}$?. En caso afirmativo, ¿cuál es el autovalor asociado?
    \end{enumerate}
\end{question}

\begin{question} % Quarteroni 5.13 ej. 2 pg. 258
    Utilizando los círculos de Gerschgorin, localice el espectro de la matriz
    \[ \bm{A} = 
        \begin{bmatrix}
            1 & 2 & -1 \\
            2 & 7 & 0 \\
            -1 & 0 & 5
        \end{bmatrix} \]
\end{question}

\begin{question} % Problemas resueltos Murcia p.22 pg 42
    Halle los autovalores y autovectores de la matriz
    \[ \bm{A} = \begin{bmatrix} -13 & 20 \\ -12 & 18 \end{bmatrix} \]
\end{question}


\begin{question} % Problemas resueltos Murcia p.25 pg 43
    Halle los autovalores y autovectores de la matriz
    \[ \bm{A} = \begin{bmatrix} 2 & 0 & 0 \\ 1 & 0 & 1 \\ 0 & 0 & 2 \end{bmatrix} \]
\end{question}

\begin{question} 
Utilice el método de las potencias para determinar el autovalor dominante y su autovector asociado de las matrices:
\begin{multicols}{3}
\begin{enumerate}[a)]
  \item 
\[ \bm{A} = \begin{bmatrix} 4 & 2 & 3 \\ 3 & 0 & 4 \\ 1 & 2 & 5 \end{bmatrix} \]
\item 
\[ \bm{B} = \begin{bmatrix} 0 & 1 & 0 & 1 & 0 \\ 1 & 0 & 1 & 0 & 1 \\ 0 & 1 & 0 & 1 & 0 \\ 1 & 0 & 1 & 0 & 1 \\ 0 & 1 & 0 & 1 & 0 \end{bmatrix} \]
\item 
    \[ \bm{C} = \begin{bmatrix}
        1 & 2 & 3 & 4 & 5 & 6 \\
        7 & 8 & 9 & 10 & 11 & 12 \\
        13 & 14 & 15 & 16 & 17 & 18 \\
        19 & 20 & 21 & 22 & 23 & 24 \\
        25 & 26 & 27 & 28 & 29 & 30 \\
        31 & 32 & 33 & 34 & 35 & 36
    \end{bmatrix} \]

\end{enumerate}
\end{multicols}
\end{question}
\end{document}

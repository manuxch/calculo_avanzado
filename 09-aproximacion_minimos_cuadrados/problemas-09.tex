\documentclass[11pt]{article}

\usepackage{minted}
\usepackage[scale=1]{ccicons}
\usepackage{metalogo}
\usepackage{xcolor,colortbl}
\usepackage{multicol,multirow,booktabs}
\usepackage{graphicx}
\usepackage{bm}
\usepackage{fontawesome}
\usepackage{exsheets}
\usepackage[paper=a4paper, headheight=110pt,showframe=false, 
            layoutvoffset=2em,
            bottom=2cm, top=3.5cm]{geometry}
\usepackage[spanish, es-nodecimaldot]{babel}
\usepackage[babel]{microtype}
\usepackage{hyperref}
\usepackage{amsmath}
\usepackage{mismath}
\usepackage{gensymb,amssymb}
\setlength{\parindent}{3em}
\setlength{\parskip}{1em} 
\usepackage[shortlabels]{enumitem}
\usepackage{subcaption}
\usepackage{wrapfig}
%\usepackage{mathspec}
\usepackage{unicode-math}


% Fonts can be customized here.
\defaultfontfeatures{Mapping=tex-text}
\setmainfont [Ligatures={Common}]{Linux Libertine O}
\setmonofont[Scale=0.9]{Linux Libertine Mono O}
%\usepackage[svgnames]{xcolor} % Gestión de colores
\usepackage{hyperref}
\hypersetup{
  colorlinks=true, linktocpage=true, pdfstartpage=3, pdfstartview=FitV,%
  breaklinks=true, pageanchor=true,%
  pdfpagemode=UseNone, %
  plainpages=false, bookmarksnumbered, bookmarksopen=true, bookmarksopenlevel=1,%
  hypertexnames=true, pdfhighlight=/O,%nesting=true,%frenchlinks,%
  urlcolor=Maroon, linkcolor=RoyalBlue, citecolor=Blue, %pagecolor=RoyalBlue,%
  pdftitle={},%
  pdfauthor={\textcopyright\ C. Manuel Carlevaro},%
  pdfsubject={},%
  pdfkeywords={},%
  pdfcreator={XeLaTeX},%
  pdfproducer={XeLaTeX}%
}

%% Operadores
\DeclareMathOperator{\sen}{sen}
\DeclareMathOperator{\senc}{senc}
\DeclareMathOperator{\sign}{sign}
\newcommand{\T}[1]{\underline{\bm{#1}}}
\DeclareMathOperator{\Tr}{Tr}
%\NewDocumentCommand{\evalat}{sO{\big}mm}{%
  %\IfBooleanTF{#1}
   %{\mleft. #3 \mright|_{#4}}
   %{#3#2|_{#4}}%
%}


\title{Cálculo avanzado}
\author{Dpto. de Ingenería Mecánica}
\date{Tema 9: Aproximación discreta y continua por mínimos cuadrados}


\begin{document}
% \maketitle

\begin{center}
\framebox[1.0\textwidth][c]{
\huge{\textsc{Cálculo Avanzado}} 
}
\end{center} 

\begin{center}
\vspace{\baselineskip}
\Large{\textsc{Departamento de Ingenería Mecánica}} \\
\textsc{Facultad Regional La Plata} \\
\textsc{Universidad Tecnológica Nacional}
\end{center}

% \vspace{1em}

\begin{center}
\begin{tabular}{r l}
    \textbf{Práctica:} & Unidad 9. \\
 \textbf{Tema:} & Aproximación discreta por mínimos cuadrados. \\
 \textbf{Profesor Titular:} & Manuel Carlevaro. \\
 \textbf{Ayudante de Primera} & Christian Molina. \\
\end{tabular}\end{center}
\vspace{1em}

\begin{question} % Bradie  Excersise 5 pg 441
El costo total de producción en función del número de horas de máquina se proporciona para una muestra de nueve procesos de producción. Estime los costos fijos y variables asociados con este proceso.

\begin{center}
    \begin{tabular}{lccccccccc}
        \toprule
        \textbf{Horas de máquina} & 22 & 23 & 19 & 12 & 12 & 9 & 7 & 11 & 14 \\
        \textbf{Costo total} (en miles \$) & 23 & 25 & 20 & 20 & 20 & 15 & 14 & 14 & 16 \\
        \bottomrule
    \end{tabular}
\end{center}
\end{question}

\begin{question} % Bradie Excersise 6 pg 441
La resistividad del platino como función de la temperatura se da en la tabla siguiente. Estime los parámetros de un ajuste lineal de los datos y prediga la resistividad cuando la temperatura es $365$ K.

\begin{center}
    \begin{tabular}{lccccccccc}
        \toprule
        \textbf{Temperatura} (K) & 100 & 200 & 300 & 400 & 500 \\
        \textbf{Resistividad} (\textohm{} cm, $\mul 10^6$) & 4.1 & 8.0 & 12.6 & 16.3 & 19.4 \\
        \bottomrule
    \end{tabular}
\end{center}
\end{question}

\begin{question} % Bradie Excersise 7 pg 441
La tabla siguiente muestra el tiempo (en segundos) requerido para descargar un tanque de agua a través de un orificio en su fondo, como función de la altura de llenado (en cm) del tanque.
\begin{center}
    \begin{tabular}{lccccccccc}
        \toprule
        \textbf{Altura} (cm) & 0.5 & 1.0 & 1.5 & 2.0 & 2.5 & 3.0 & 3.5 & 4.0 \\
        \textbf{Tiempo} (s) & 65.99 & 120.28 & 166.69 & 207.85 & 245.41 & 279.95 & 313.04 & 344.24 \\
        \bottomrule
    \end{tabular}
\end{center}
\begin{enumerate}[a)]
    \item Construir un gráfico de estos datos. ¿Cuál es la forma funcional más apropiada para ajustar los datos?
    \item Ajuste los datos para la función indicada en la parte a).
\end{enumerate}
\end{question}

\begin{question} % Bradie Excersise 9 pg 441
La presión barométrica, como función de la altura sobre el nivel del mar, $h$, se modela con la relación $P = \alpha e^{-\beta h}$. Use los datos de la tabla siguiente para estimar los parámtros del model y predecir la presión barométrica a una altura de 
\begin{center}
    \begin{tabular}{lccccccccc}
        \toprule
        \textbf{Presión barométrica} (mmHg) & 29.9 & 29.4 & 29.0 & 28.4 & 27.7 \\
        \textbf{Altura} (m) & 0.0 & 12.7 & 25.4 & 38.1 & 50.8 \\
        \bottomrule
    \end{tabular}
\end{center}
\end{question}

\begin{question} % Burden ej 8.2 1) a, b, d y e pg 185
Encuentre la aproximación lineal por mínimos cuadrados para $f(x)$ en el intervalo indicado si:
\begin{enumerate}[a)]
    \item $f(x) = x^2 + 3 x + 2$, en $[0, 1]$
    \item $f(x) = x^3$ en $[0, 2]$
    \item $f(x) = e^x$, en $[0, 2]$
    \item $f(x) = \frac{1}{2} \cos x + \frac{1}{3} \sen x$, en $[0, 1]$
\end{enumerate}
\end{question}

\begin{question} % Burden ej 8.2 3) pg 186
    Encuentre la aproximación polinomial por mínimos cuadrados de grado 2 para las funciones e intervalos del Ejercicio 1.
\end{question}

\begin{question} % Burden ej 8.2 2) pg 186
    Encuentre la aproximación polinomial por mínimos cuadrados de grado 2 para las funciones del Ejercicio 1 en el intervalo $[-1, 1]$.
\end{question}

\begin{question} % Epperson Example 4.14 pg 244 
    Construya una aproximación de mínimos cuadrados de cuarto grado a la función exponencial
    \[ f(x) = e^x \]
    sobre el intervalo $[-1, 1]$ usando polinomios de Legedre.
\end{question}

\begin{question} % Moreno Gonzalez. Ej. 40 pg. 119
Determinar la mejor aproximación a $x^3$ con un polinomio de segundo grado usando polinomios de Chebishev.
\end{question}


\begin{question} % Moreno Gonzalez. Ej. 42 pg. 120
Hallar la recta que mejor aproxima la gráfica de la función 
\[ f(x) = \frac{1}{1 + x^2} \]
con la norma inducida por el producto interno:
\[ \langle f, g \rangle = \int_0^5 f(x) g(x) \, dx \] 
usando una base de polinomios ortogonales.
\end{question}
\end{document}

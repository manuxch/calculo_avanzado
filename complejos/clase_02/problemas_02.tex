\documentclass[11pt]{article}

\usepackage{fontspec}
\usepackage{minted}
\usepackage[scale=1]{ccicons}
\usepackage{metalogo}
\usepackage{xcolor,colortbl}
\usepackage{multicol,multirow,booktabs}
\usepackage{graphicx}
\usepackage{bm}
\usepackage{fontawesome}
\usepackage{exsheets}
\usepackage[paper=a4paper, headheight=110pt,showframe=false,
	layoutvoffset=2em,
	bottom=2cm, top=3.5cm]{geometry}
\usepackage[spanish, es-nodecimaldot]{babel}
\usepackage[babel]{microtype}
\usepackage{hyperref}
\usepackage{amsmath}
\usepackage{mismath}
\usepackage{mathrsfs}
\usepackage{gensymb,amssymb}
\setlength{\parindent}{3em}
\setlength{\parskip}{1em}
\usepackage[shortlabels]{enumitem}
\usepackage{subcaption}
\usepackage{wrapfig}
\usepackage[svgnames]{xcolor} % Gestión de colores
%\usepackage{mathspec}
% \usepackage{unicode-math}


% Fonts can be customized here.
\setmainfont[Ligatures=TeX]{Linux Libertine O}
\setmonofont[Scale=0.90,Ligatures=TeX]{Hack Nerd Font Mono}
\usepackage{hyperref}
\hypersetup{
	colorlinks=true, linktocpage=true, pdfstartpage=3, pdfstartview=FitV,%
	breaklinks=true, pageanchor=true,%
	pdfpagemode=UseNone, %
	plainpages=false, bookmarksnumbered, bookmarksopen=true, bookmarksopenlevel=1,%
	hypertexnames=true, pdfhighlight=/O,%nesting=true,%frenchlinks,%
	urlcolor=Maroon, linkcolor=RoyalBlue, citecolor=Blue, %pagecolor=RoyalBlue,%
	pdftitle={},%
	pdfauthor={\textcopyright\ C. Manuel Carlevaro},%
	pdfsubject={},%
	pdfkeywords={},%
	pdfcreator={XeLaTeX}%
}

%% Operadores
\DeclareMathOperator{\sen}{sen}
\DeclareMathOperator{\senc}{senc}
\DeclareMathOperator{\sign}{sign}
\newcommand{\T}[1]{\underline{\bm{#1}}}
\DeclareMathOperator{\Tr}{Tr}
%\NewDocumentCommand{\evalat}{sO{\big}mm}{%
%\IfBooleanTF{#1}
%{\mleft. #3 \mright|_{#4}}
%{#3#2|_{#4}}%
%}


\title{Cálculo avanzado}
\author{Dpto. de Ingenería Mecánica}
\date{Clase 2: números complejos}


\begin{document}

\begin{center}
\framebox[1.0\textwidth][c]{
\huge{\textsc{Cálculo Avanzado}} 
}
\end{center} 

\begin{center}
\vspace{\baselineskip}
\Large{\textsc{Departamento de Ingenería Mecánica}} \\
\textsc{Facultad Regional La Plata} \\
\textsc{Universidad Tecnológica Nacional}
\end{center}

% \vspace{1em}
%%%% Formato de problemas:
% Cada problema tiene el formato:
% \begin{question} % Referencia del problema
%   \begin{enumerate}[a)] % En caso que haya ítems del problema
%   \end{enumerate}
% \end{question}
%%%%

\begin{center}
\begin{tabular}{r l}
    \textbf{Práctica:} & 2 \\
 \textbf{Tema:} & Introducción a la variable compleja. \\
 \textbf{Profesor Titular:} & Manuel Carlevaro \\
 \textbf{Jefe de Trabajos Prácticos:} & Diego Amiconi \\
 \textbf{Ayudante de Primera:} & Lucas Basiuk 
\end{tabular}\end{center}

\vspace{1em}

\begin{question} % H. Gross - 1.8.1(L) pg 3.
    \begin{enumerate}[a)]
        \item Calcular:
        \[ \int_0^{2i} z dz \]
    \item Calcular:
        \[ \int_0^{2 i} \bar{z} dz \]
\end{enumerate}
primero a lo largo del segmento de línea $C_1$ que une $0$ con $2i$, y luego a lo largo de la curva $C_2$, donde $C_2$ es la mitad derecha del círculo centrado en $i$ con radio $1$.
\end{question}

\begin{question} % H. Gross - 1.8.2 pg 3.
Explicar por qué la integral:
\[ \int_1^i 2 e^{2z} \, dz \]
no es ambigua, y encontrar el valor de esta integral.
\end{question}

\begin{question}  % H. Gross - 1.8.3 pg 3.
    Calcular:
    \[ \int_1^i \bar{z}^2 \, dz \]
    a lo largo de las siguientes curvas $C$:
    \begin{enumerate}[a)]
        \item $C$ es el segmento de línea que une $1$ con $i$.
        \item $C = \{z : z = e^{i \theta}, \; 0 \leq \theta \leq \dfrac{\pi}{2}\} $, es decir, $C$ es el primer cuadrante del círculo $|z| = 1$.
    \end{enumerate}
\end{question}

\begin{question}% H. Gross - 1.7.1(L) pg 3.
    Suponga que $\lim_{n \rightarrow \infty} a_n = L_1$ y $\lim_{n \rightarrow \infty} a_n = L_2$. Probar que $L_1 = L_2$.
\end{question}


\end{document}

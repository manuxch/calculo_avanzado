\documentclass[11pt]{article}

\usepackage{fontspec}
\usepackage{minted}
\usepackage[scale=1]{ccicons}
\usepackage{metalogo}
\usepackage{xcolor,colortbl}
\usepackage{multicol,multirow,booktabs}
\usepackage{graphicx}
\usepackage{bm}
\usepackage{fontawesome}
\usepackage{exsheets}
\usepackage[paper=a4paper, headheight=110pt,showframe=false,
	layoutvoffset=2em,
	bottom=2cm, top=3.5cm]{geometry}
\usepackage[spanish, es-nodecimaldot]{babel}
\usepackage[babel]{microtype}
\usepackage{hyperref}
\usepackage{amsmath}
\usepackage{mismath}
\usepackage{mathrsfs}
\usepackage{gensymb,amssymb}
\setlength{\parindent}{3em}
\setlength{\parskip}{1em}
\usepackage[shortlabels]{enumitem}
\usepackage{subcaption}
\usepackage{wrapfig}
\usepackage[svgnames]{xcolor} % Gestión de colores
%\usepackage{mathspec}
% \usepackage{unicode-math}


% Fonts can be customized here.
\setmainfont[Ligatures=TeX]{Linux Libertine O}
\setmonofont[Scale=0.90,Ligatures=TeX]{Hack Nerd Font Mono}
\usepackage{hyperref}
\hypersetup{
	colorlinks=true, linktocpage=true, pdfstartpage=3, pdfstartview=FitV,%
	breaklinks=true, pageanchor=true,%
	pdfpagemode=UseNone, %
	plainpages=false, bookmarksnumbered, bookmarksopen=true, bookmarksopenlevel=1,%
	hypertexnames=true, pdfhighlight=/O,%nesting=true,%frenchlinks,%
	urlcolor=Maroon, linkcolor=RoyalBlue, citecolor=Blue, %pagecolor=RoyalBlue,%
	pdftitle={},%
	pdfauthor={\textcopyright\ C. Manuel Carlevaro},%
	pdfsubject={},%
	pdfkeywords={},%
	pdfcreator={XeLaTeX}%
}

%% Operadores
\DeclareMathOperator{\sen}{sen}
\DeclareMathOperator{\senc}{senc}
\DeclareMathOperator{\sign}{sign}
\newcommand{\T}[1]{\underline{\bm{#1}}}
\DeclareMathOperator{\Tr}{Tr}
%\NewDocumentCommand{\evalat}{sO{\big}mm}{%
%\IfBooleanTF{#1}
%{\mleft. #3 \mright|_{#4}}
%{#3#2|_{#4}}%
%}


\title{Cálculo avanzado}
\author{Dpto. de Ingenería Mecánica}
\date{Clase 1: números complejos}


\begin{document}
% \maketitle

\begin{center}
\framebox[1.0\textwidth][c]{
\huge{\textsc{Cálculo Avanzado}} 
}
\end{center} 

\begin{center}
\vspace{\baselineskip}
\Large{\textsc{Departamento de Ingenería Mecánica}} \\
\textsc{Facultad Regional La Plata} \\
\textsc{Universidad Tecnológica Nacional}
\end{center}

% \vspace{1em}

\begin{center}
\begin{tabular}{r l}
 \textbf{Tema:} & Introducción a la variable compleja. \\
 \textbf{Profesor Titular:} & Manuel Carlevaro \\
 \textbf{Jefe de Trabajos Prácticos:} & Diego Amiconi \\
 \textbf{Ayudante de Primera:} & Lucas Basiuk 
\end{tabular}\end{center}

\vspace{1em}

\begin{question} % Kreyszig PS 13.1 - 1 pg 613
 Mostrar que $i^2 = -1$, $i^3 = -i$, $i^4 = 1$, $i^5 = i$, y $1/i = -i$, $1/i^2 = -1$, $1/i^3 = -i$.
\end{question}

\begin{question} % Kreyszig PS 13.1 - 2 pg 613
 Multiplicar por $i$ equivale geométricamente a rotar en sentido antihorario por $\pi/2$ ($90\degree$). Verificar graficando $z$ y $zi$, y el ángulo de rotación, para $z = 1 + i$, $z = -1 + 2 i$, $z = 4 - 3 i$.
\end{question}

\begin{question} % Kreyszig PS 13.1 - 4 pg 614
 Verificar las siguientes propiedades de los números complejos conjugados:
 \begin{align*}
  \overline{(z_1 + z_2)} = \overline{z_1} + \overline{z_2} &\qquad \overline{(z_1 - z_2)} = \overline{z_1} - \overline{z_2} \\
  \overline{(z_1 z_2)} = \overline{z_1} \; \overline{z_2} &\qquad \overline{ \left( \frac{z_1}{z_2} \right) } = \frac{\overline{z_1}}{\overline{z_2}} \\
 \end{align*}
 para $z_1 = -11 + 10 i$ y $z_2 = -1 + 4 i$.
\end{question}

\begin{question} % HGross Lec.1 - 1 pg. 12
    Expresar $\dfrac{3 + 5 i}{7 + 9 i}$ en la forma $a + bi$, donde $a$ y $b$ son reales.
\end{question}

\begin{question} % HGross Lec.1 - 2 pg. 12
    En términos del diagrama de Argand, describir la región de puntos definida por:
    \[ \begin{cases}
        |z - (1 + i)| < 2 \\
        |z - 2 i| > \dfrac{3}{2}
       \end{cases}
\]
\end{question}

\begin{question} % HGross Lec.1 - 3 pg. 12
    \begin{enumerate}[a)]
     \item En términos del diagrama de Argand, describir el conjunto:
     \[ S = \{ z:z = \cos t + i \sen t, 0 \leq t \leq \pi \} \].
     \item Describir $f(S)$ si $f$ se define como $f(z) = z^2$.
    \end{enumerate}
\end{question}

\begin{question} % Kreyszig PS 13.2 - 13 pg 618
 Determinar el valor principal del argumento de $(1 + i)^{20}$.
\end{question}

\begin{question} % Kreyszig PS 13.2 - 21 pg 618
 Encontrar y graficar en el plano complejo todas las raíces de $\sqrt[3]{i + i}$.
\end{question}

\begin{question} % Kreyszig PS 13.3 - 11 pg 624
    Determinar $\Re(f)$ e $\Im(f)$ para
    \[ f(z ) = \frac{1}{1 - z} \]
    en $z = 1 - i$.
\end{question}

\begin{question} % Kreyszig PS 13.3 - 17 pg 624
    Del mismo modo que para las funciones de variable real, una función compleja de variable compleja es \textit{continua} en $z = z_0$ si $f(z_0)$ está definida y
    \[ \lim_{z \rightarrow z_0} f(z) = f(z_0) \]
    
    Determinar si $f(z)$ es continua en $z = 0$, si $f(0) = 0$ y para $z \neq 0$ la función se define como
    \[ f(z) = \begin{cases}
                \frac{\Re(z)}{1 - |z|}, &\quad z \neq 0 \\
                0, &\quad z = 0
              \end{cases} \]
 \end{question}

\begin{question} % HGross Lec.1 - 4 pg. 12
    Si $f(z) = z^3$, escribir $f$ en la forma $u(x, y) + i v(x,y)$ y mostrar que $u$ y $v$ satisfacen las condiciones de Cauchy-Riemann.
\end{question}

\begin{question} % Kreyszig PS 13.4 - 31 pg 629
    Determinar si las funciones
    \begin{enumerate}[a)]
     \item \[ f(z ) = e^{-2x} (\cos 2y - i \sen 2y) \]
     \item \[ f(z) = \Re(z^2) - i \Im(z^2) \]
    \end{enumerate}
    son analíticas.
\end{question}

\begin{question} % HGross Lec.1 - 5 pg. 12
    Si $u(x, y) = 3 x - 2 y + 5$, ¿cómo debe estar definida $v(x, y)$ si $u(x, y) + i v(x, y)$ debe ser analítica?
\end{question}


\end{document}

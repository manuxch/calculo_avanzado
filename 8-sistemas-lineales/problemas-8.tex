\documentclass[11pt]{article}

\usepackage{minted}
\usepackage[scale=1]{ccicons}
\usepackage{metalogo}
\usepackage{xcolor,colortbl}
\usepackage{multicol,multirow,booktabs}
\usepackage{graphicx}
\usepackage{bm}
\usepackage{fontawesome}
\usepackage{exsheets}
\usepackage[paper=a4paper, headheight=110pt,showframe=false, 
            layoutvoffset=2em,
            bottom=2cm, top=3.5cm]{geometry}
\usepackage[spanish, es-nodecimaldot]{babel}
\usepackage[babel]{microtype}
\usepackage{hyperref}
\usepackage{amsmath}
\usepackage{mismath}
\usepackage{gensymb,amssymb}
\setlength{\parindent}{3em}
\setlength{\parskip}{1em} 
\usepackage[shortlabels]{enumitem}
\usepackage{subcaption}
\usepackage{wrapfig}
%\usepackage{mathspec}
\usepackage{unicode-math}


% Fonts can be customized here.
\defaultfontfeatures{Mapping=tex-text}
\setmainfont [Ligatures={Common}]{Linux Libertine O}
\setmonofont[Scale=0.9]{Linux Libertine Mono O}
%\usepackage[svgnames]{xcolor} % Gestión de colores
\usepackage{hyperref}
\hypersetup{
  colorlinks=true, linktocpage=true, pdfstartpage=3, pdfstartview=FitV,%
  breaklinks=true, pageanchor=true,%
  pdfpagemode=UseNone, %
  plainpages=false, bookmarksnumbered, bookmarksopen=true, bookmarksopenlevel=1,%
  hypertexnames=true, pdfhighlight=/O,%nesting=true,%frenchlinks,%
  urlcolor=Maroon, linkcolor=RoyalBlue, citecolor=Blue, %pagecolor=RoyalBlue,%
  pdftitle={},%
  pdfauthor={\textcopyright\ C. Manuel Carlevaro},%
  pdfsubject={},%
  pdfkeywords={},%
  pdfcreator={XeLaTeX},%
  pdfproducer={XeLaTeX}%
}

%% Operadores
\DeclareMathOperator{\sen}{sen}
\DeclareMathOperator{\senc}{senc}
\DeclareMathOperator{\sign}{sign}
\newcommand{\T}[1]{\underline{\bm{#1}}}
\DeclareMathOperator{\Tr}{Tr}
%\NewDocumentCommand{\evalat}{sO{\big}mm}{%
  %\IfBooleanTF{#1}
   %{\mleft. #3 \mright|_{#4}}
   %{#3#2|_{#4}}%
%}


\title{Cálculo avanzado}
\author{Dpto. de Ingenería Mecánica}
\date{Clase 11: Sistemas de ecuaciones lineales}


\begin{document}
% \maketitle

\begin{center}
\framebox[1.0\textwidth][c]{
\huge{\textsc{Cálculo Avanzado}} 
}
\end{center} 

\begin{center}
\vspace{\baselineskip}
\Large{\textsc{Departamento de Ingenería Mecánica}} \\
\textsc{Facultad Regional La Plata} \\
\textsc{Universidad Tecnológica Nacional}
\end{center}

% \vspace{1em}

\begin{center}
\begin{tabular}{r l}
    \textbf{Práctica:} & 8 \\
 \textbf{Tema:} & Sistemas de ecuaciones lineales. \\
 \textbf{Profesor Titular:} & Manuel Carlevaro \\
 \textbf{Jefe de Trabajos Prácticos:} & Diego Amiconi \\
\end{tabular}\end{center}

\vspace{1em}

\begin{question} % Bradie Problemas 1-5 pg 171
Escriba la matriz aumentada para los siguientes sistemas lineales de ecuaciones, y obtenga la solución usando eliminación gaussiana con sustitución hacia atrás:
\begin{enumerate}[a)]
    \item \[ \begin{cases} 2 x_1 - x_2 + x_3 = -1 \\
                       4 x_1 + 2 x_2 + x_3 = 4 \\ 
                       6 x_1 - 4 x_2 + 2 x_3 = -2 
        \end{cases} \]
    \item \[ \begin{cases} x_1 + 2 x_2 - x_3 = 1 \\
                      2 x_1 - x_2 + x_3 = 3 \\ 
                      -x_1 + 2 x_2 + 3 x_3 = 7 
        \end{cases} \]
    \item \[ \begin{cases} x_2 + x_3 + x_4 = 0 \\
                      3 x_1 + 3 x_3 - 4 x_4 = 7 \\ 
                      x_1 + x_2 + x_3 + 2 x_4 = 6 \\
                      2 x_1 - 3 x_2 + x_3 + 3 x_4 = 6
        \end{cases} \]
\end{enumerate}
\end{question}


\begin{question} % Bradie Problemas 10 pg 172
\begin{enumerate}[a)]
\item Resuelva el sistema:
    \[ \begin{cases}
        3.02 x_1 - 1.05 x_2 + 2.53 x_3 = -1.61 \\
        4.33 x_1 + 0.56 x_2 - 1.78 x_3 = 7.23 \\
        -0.83 x_1 - 0.54 x_2 + 1.47 x_3 = -3.38
    \end{cases} \]
    utilizando eliminación gaussiana con sustitución hacia atrás.
\item Cambie el coeficiente de $x_1$ en la primera ecuación a $3.01$ y resuelva el sistema resultante. ¿En qué porcentaje cambian las componentes del nuevo vector solución?
\item Vuelva el coeficiente de $x_1$ a su valor original en la primera ecuación, pero cambie el término independiente de la segunda ecuación a $1.99$ y resuelva el nuevo sistema. ¿Cuál es el cambio porcentual en las tres componentes de la solución comparados con sus valores de la parte a)?
\end{enumerate}
\end{question}

\begin{question} % Bradie pg 206
    Verifique que las matrices triangulares $\mathbb{L}$ y $\mathbb{U}$ siguientes:
    \[ \mathbb{L} = \begin{bmatrix} 1 & 0 & 0 \\ 2 & 1 & 0 \\ 5 & 12 & 1 \end{bmatrix}, \quad \mathbb{U} = \begin{bmatrix} 1 & 4 & 3 \\ 0 & -1 & 3 \\ 0 & 0 & -53 \end{bmatrix} \]
    factorizan la matriz
    \[ \mathbb{A} = \begin{bmatrix} 1 & 4 & 3 \\ 2 & 7 & 9 \\ 5 & 8 & -2 \end{bmatrix} \]
\end{question}

\begin{question} % Bradie Ejercicio 11 pg 217
    Resuelva el sistema $\mathbb{A} \bm{x} = \bm{b}$ para cada uno de los siguientes términos independientes:
    \[ \mathbb{A} = \begin{bmatrix} 1 & 2 & 3 & 4 \\ -1 & 1 & 2 & 3 \\ 1 & -1 & 1 & 2 \\ -1 & 1 & -1 & 5 \end{bmatrix}, \, \bm{b}_1 = \begin{bmatrix} 10 \\ 5 \\ 3 \\ 4 \end{bmatrix}, \, \bm{b}_2 = \begin{bmatrix} -4 \\ -5 \\ -3 \\ -4 \end{bmatrix}, \, \bm{b}_3 = \begin{bmatrix} -2 \\ -3 \\ 1 \\ -8 \end{bmatrix} \]
\end{question}

\begin{question} % Burden Ejemplo 3 pg 295
Resuelva el sistema lineal con aritmética de redondeo de tres dígitos y utilice una estrategia de pivoteo parcial escalado.
\[ \begin{cases}
2.11 x_1 - 4.21 x_2 + 0.921 x_3 + 2.01 \\
4.01 x_1 + 10.2 x_2 - 1.12 x_3 = -3.09 \\
1.09 x_1 + 0.987 x_2 + 0.832 x_3 = 4.21
\end{cases} \]
\end{question}

\begin{question}
Dado el siguiente sistema de ecuaciones:
\[ \begin{system}
    2 x_1 + 5 x_2 + x_3 + 8 x_4 = 78 \\
    7 x_1 + 2 x_3 + x_4 = 24 \\
    x_1 + 4 x_2 + x_3 + 4 x_4 = 46 \\
    8 x_1 + 3 x_2 + 2 x_3 + 5 x_4 = 62
\end{system} \]
resolver el sistema implementando el método de Jacobi y el de Gauss-Seidel, con relajación, y comparar las velocidades de convergencia de cada método.
\end{question}

\end{document}

\documentclass[11pt]{article}

\usepackage{fontspec}
\usepackage{minted}
\usepackage[scale=1]{ccicons}
\usepackage{metalogo}
\usepackage{xcolor,colortbl}
\usepackage{multicol,multirow,booktabs}
\usepackage{graphicx}
\usepackage{bm}
\usepackage{fontawesome}
\usepackage{exsheets}
\usepackage[paper=a4paper, headheight=110pt,showframe=false,
	layoutvoffset=2em,
	bottom=2cm, top=3.5cm]{geometry}
\usepackage[spanish, es-nodecimaldot]{babel}
\usepackage[babel]{microtype}
\usepackage{hyperref}
\usepackage{amsmath}
\usepackage{mismath}
\usepackage{mathrsfs}
\usepackage{gensymb,amssymb}
\setlength{\parindent}{3em}
\setlength{\parskip}{1em}
\usepackage[shortlabels]{enumitem}
\usepackage{subcaption}
\usepackage{wrapfig}
\usepackage[svgnames]{xcolor} % Gestión de colores
%\usepackage{mathspec}
% \usepackage{unicode-math}


% Fonts can be customized here.
\setmainfont[Ligatures=TeX]{Linux Libertine O}
\setmonofont[Scale=0.90,Ligatures=TeX]{Hack Nerd Font Mono}
\usepackage{hyperref}
\hypersetup{
	colorlinks=true, linktocpage=true, pdfstartpage=3, pdfstartview=FitV,%
	breaklinks=true, pageanchor=true,%
	pdfpagemode=UseNone, %
	plainpages=false, bookmarksnumbered, bookmarksopen=true, bookmarksopenlevel=1,%
	hypertexnames=true, pdfhighlight=/O,%nesting=true,%frenchlinks,%
	urlcolor=Maroon, linkcolor=RoyalBlue, citecolor=Blue, %pagecolor=RoyalBlue,%
	pdftitle={},%
	pdfauthor={\textcopyright\ C. Manuel Carlevaro},%
	pdfsubject={},%
	pdfkeywords={},%
	pdfcreator={XeLaTeX}%
}

%% Operadores
\DeclareMathOperator{\sen}{sen}
\DeclareMathOperator{\senc}{senc}
\DeclareMathOperator{\sign}{sign}
\newcommand{\T}[1]{\underline{\bm{#1}}}
\DeclareMathOperator{\Tr}{Tr}
%\NewDocumentCommand{\evalat}{sO{\big}mm}{%
%\IfBooleanTF{#1}
%{\mleft. #3 \mright|_{#4}}
%{#3#2|_{#4}}%
%}


\title{Cálculo avanzado}
\author{Dpto. de Ingenería Mecánica}
\date{Clase 4: integral y transformada de Fourier}


\begin{document}
% \maketitle

\begin{center}
\framebox[1.0\textwidth][c]{
\huge{\textsc{Cálculo Avanzado}} 
}
\end{center} 

\begin{center}
\vspace{\baselineskip}
\Large{\textsc{Departamento de Ingenería Mecánica}} \\
\textsc{Facultad Regional La Plata} \\
\textsc{Universidad Tecnológica Nacional}
\end{center}

% \vspace{1em}

\begin{center}
\begin{tabular}{r l}
    \textbf{Práctica:} & 3 \\
 \textbf{Tema:} & Funciones ortogonales. Series de Fourier. \\
 \textbf{Profesor Titular:} & Manuel Carlevaro \\
 \textbf{Jefe de Trabajos Prácticos:} & Diego Amiconi \\
 \textbf{Ayudante de Primera:} & Lucas Basiuk 
\end{tabular}\end{center}

\vspace{1em}

\begin{question} % Kreyszig problem set 11.7 - 1 pg. 517
Halle la representación integral de Fourier de la función $f(x)$ dada por:
\[ f(x) = 
    \begin{cases}
    0 & \text{ si } x < 0 \\
    \dfrac{\pi}{2} & \text{ si } x = 0 \\
    \pi e^{-x} & \text{ si } x > 0 \\
\end{cases} \]
\end{question}


\begin{question} % Kreyszig problem set 11.8 - 1 pg. 522
Obtenga la transformada coseno de Fourier $\hat{f}_c(\omega)$ de la función
\[ f(x) = 
\begin{cases}
    1 & \text{ si } 0 < x < 1 \\
    -1 & \text{ si } 1 < x < 2 \\
    0 & \text{ si } x > 2 
\end{cases} \]
\end{question}


\begin{question} % Kreyszig problem set 11.8 - 11 pg. 522
Obtenga la transformada seno de Fourier $\hat{f}_s(\omega)$ de la función
\[ f(x) = 
\begin{cases}
    x^2 & \text{ si } 0 < x < 1 \\
    0 & \text{ si } x > 1 \\
\end{cases} \]
\end{question}



\end{document}

\documentclass[11pt]{article}

\usepackage{minted}
\usepackage[scale=1]{ccicons}
\usepackage{metalogo}
\usepackage{xcolor,colortbl}
\usepackage{multicol,multirow,booktabs}
\usepackage{graphicx}
\usepackage{bm}
\usepackage{fontawesome}
\usepackage{exsheets}
\usepackage[paper=a4paper, headheight=110pt,showframe=false, 
            layoutvoffset=2em,
            bottom=2cm, top=3.5cm]{geometry}
\usepackage[spanish, es-nodecimaldot]{babel}
\usepackage[babel]{microtype}
\usepackage{hyperref}
\usepackage{amsmath}
\usepackage{mismath}
\usepackage{gensymb,amssymb}
\setlength{\parindent}{3em}
\setlength{\parskip}{1em} 
\usepackage[shortlabels]{enumitem}
\usepackage{subcaption}
\usepackage{wrapfig}
%\usepackage{mathspec}
\usepackage{unicode-math}


% Fonts can be customized here.
\defaultfontfeatures{Mapping=tex-text}
\setmainfont [Ligatures={Common}]{Linux Libertine O}
\setmonofont[Scale=0.9]{Linux Libertine Mono O}
%\usepackage[svgnames]{xcolor} % Gestión de colores
\usepackage{hyperref}
\hypersetup{
  colorlinks=true, linktocpage=true, pdfstartpage=3, pdfstartview=FitV,%
  breaklinks=true, pageanchor=true,%
  pdfpagemode=UseNone, %
  plainpages=false, bookmarksnumbered, bookmarksopen=true, bookmarksopenlevel=1,%
  hypertexnames=true, pdfhighlight=/O,%nesting=true,%frenchlinks,%
  urlcolor=Maroon, linkcolor=RoyalBlue, citecolor=Blue, %pagecolor=RoyalBlue,%
  pdftitle={},%
  pdfauthor={\textcopyright\ C. Manuel Carlevaro},%
  pdfsubject={},%
  pdfkeywords={},%
  pdfcreator={XeLaTeX},%
  pdfproducer={XeLaTeX}%
}

%% Operadores
\DeclareMathOperator{\sen}{sen}
\DeclareMathOperator{\senc}{senc}
\DeclareMathOperator{\sign}{sign}
\newcommand{\T}[1]{\underline{\bm{#1}}}
\DeclareMathOperator{\Tr}{Tr}
%\NewDocumentCommand{\evalat}{sO{\big}mm}{%
  %\IfBooleanTF{#1}
   %{\mleft. #3 \mright|_{#4}}
   %{#3#2|_{#4}}%
%}


\title{Cálculo avanzado}
\author{Dpto. de Ingenería Mecánica}


\begin{document}
% \maketitle

\begin{center}
\framebox[1.0\textwidth][c]{
\huge{\textsc{Cálculo Avanzado}} 
}
\end{center} 

\begin{center}
\vspace{\baselineskip}
\Large{\textsc{Departamento de Ingenería Mecánica}} \\
\textsc{Facultad Regional La Plata} \\
\textsc{Universidad Tecnológica Nacional}
\end{center}

% \vspace{1em}

\begin{center}
\begin{tabular}{r l}
    \textbf{Práctica:} & Unidad 12. \\
 \textbf{Tema:} & Ecuaciones en derivadas parciales. \\
 \textbf{Profesor Titular:} & Manuel Carlevaro. \\
 \textbf{Ayudante de Primera:} & Christian Molina. \\
\end{tabular}\end{center}

\vspace{1em}

\begin{question} % Burden y Faires, Numerical Analysis, Ex.Set 12.1 p1 pg 723 
Aproxime utilizando diferencias finitas la ecuación en derivadas parciales elíptica
\[ \frac{\partial^2 u}{\partial x^2} +  \frac{\partial^2 u}{\partial y^2} = 4, \quad 0 < x < 1, \; 0 < y < 2 \]
con las condiciones de frontera:
\begin{align*}
    u(x, 0) &= x^2,\; u(x, 2) = (x - 2)^2,\; 0 \leq x \leq 1 \\
    u(0, y) &= y^2,\; u(1, y) = (y - 1)^2,\; 0 \leq y \leq 2 \\
\end{align*}
Use $h = k = 1/2$ y compare los resultados con la solución exacta $u(x, y) = (x - y)^2$.
\end{question}

\begin{question} % Burden y Faires, Numerical Analysis, Ex.Set 12.1 p2 pg 723 
Aproxime utilizando diferencias finitas la ecuación en derivadas parciales elíptica
\[ \frac{\partial^2 u}{\partial x^2} +  \frac{\partial^2 u}{\partial y^2} = 0, \quad 1 < x < 2, \; 0 < y < 1 \]
con las condiciones de frontera:
\begin{align*}
    u(x, 0) &= 2 \ln x,\; u(x, 1) = \ln(x^2 + 1),\; 1 \leq x \leq 2 \\
    u(1, y) &= \ln(y^2 + 1),\; u(2, y) = \ln(y^2 + 4),\; 0 \leq y \leq 1 \\
\end{align*}
Use $h = k = 1/3$ y compare los resultados con la solución exacta $u(x, y) = \ln(x^2 + y^2)$.
\end{question}

\begin{question} % Burden y Faires, Numerical Analysis, Ex.Set 12.2 p1 pg 723 
Aproxime utilizando diferencias finitas hacia adelante la ecuación en derivadas parciales parabólica
\[ \frac{\partial u}{\partial t} - \frac{\partial^2 u}{\partial x^2} = 0, \quad 0 < x < 2, \; 0 < t < T \]
con las condiciones de frontera:
\begin{equation*}
    u(0, t) = u(2, t) = 0,\;0 < t; \; u(x, 0) = \sen \frac{\pi}{2} x, \; 0 \leq x \leq 2 
\end{equation*}
Use $n = 2$ (puntos en $x$), $T = 0.1$ y $m = 4$ (divisiones de $t$). Compare los resultados con la solución exacta $u(x, t) = \exp[-(\pi^2/4) t] \sen(\pi x/ 2)$.
\end{question}


\end{document}

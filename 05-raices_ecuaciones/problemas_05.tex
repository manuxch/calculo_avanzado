\documentclass[11pt]{article}

\usepackage{minted}
\usepackage[scale=1]{ccicons}
\usepackage{metalogo}
\usepackage{xcolor,colortbl}
\usepackage{multicol,multirow,booktabs}
\usepackage{graphicx}
\usepackage{bm}
\usepackage{fontawesome}
\usepackage{exsheets}
\usepackage[paper=a4paper, headheight=110pt,showframe=false, 
            layoutvoffset=2em,
            bottom=2cm, top=3.5cm]{geometry}
\usepackage[spanish, es-nodecimaldot]{babel}
\usepackage[babel]{microtype}
\usepackage{hyperref}
\usepackage{amsmath}
\usepackage{mismath}
\usepackage{gensymb,amssymb}
\setlength{\parindent}{3em}
\setlength{\parskip}{1em} 
\usepackage[shortlabels]{enumitem}
\usepackage{subcaption}
\usepackage{wrapfig}
%\usepackage{mathspec}
\usepackage{unicode-math}


% Fonts can be customized here.
\defaultfontfeatures{Mapping=tex-text}
\setmainfont [Ligatures={Common}]{Linux Libertine O}
\setmonofont[Scale=0.9]{Linux Libertine Mono O}
%\usepackage[svgnames]{xcolor} % Gestión de colores
\usepackage{hyperref}
\hypersetup{
  colorlinks=true, linktocpage=true, pdfstartpage=3, pdfstartview=FitV,%
  breaklinks=true, pageanchor=true,%
  pdfpagemode=UseNone, %
  plainpages=false, bookmarksnumbered, bookmarksopen=true, bookmarksopenlevel=1,%
  hypertexnames=true, pdfhighlight=/O,%nesting=true,%frenchlinks,%
  urlcolor=Maroon, linkcolor=RoyalBlue, citecolor=Blue, %pagecolor=RoyalBlue,%
  pdftitle={},%
  pdfauthor={\textcopyright\ C. Manuel Carlevaro},%
  pdfsubject={},%
  pdfkeywords={},%
  pdfcreator={XeLaTeX},%
  pdfproducer={XeLaTeX}%
}

%% Operadores
\DeclareMathOperator{\sen}{sen}
\DeclareMathOperator{\senc}{senc}
\DeclareMathOperator{\sign}{sign}
\newcommand{\T}[1]{\underline{\bm{#1}}}
\DeclareMathOperator{\Tr}{Tr}
%\NewDocumentCommand{\evalat}{sO{\big}mm}{%
  %\IfBooleanTF{#1}
   %{\mleft. #3 \mright|_{#4}}
   %{#3#2|_{#4}}%
%}


\title{Cálculo avanzado}
\author{Dpto. de Ingenería Mecánica}


\begin{document}
% \maketitle

\begin{center}
\framebox[1.0\textwidth][c]{
\huge{\textsc{Cálculo Avanzado}} 
}
\end{center} 

\begin{center}
\vspace{\baselineskip}
\Large{\textsc{Departamento de Ingenería Mecánica}} \\
\textsc{Facultad Regional La Plata} \\
\textsc{Universidad Tecnológica Nacional}
\end{center}

% \vspace{1em}

\begin{center}
\begin{tabular}{r l}
    \textbf{Práctica:} & Unidad 5 \\
 \textbf{Tema:} & Cálculo de raíces: soluciones de ecuaciones de una variable. \\
 \textbf{Profesor Titular:} & Manuel Carlevaro. \\
 \textbf{Ayudante de Primera:} & Christian Molina.
\end{tabular}\end{center}

\vspace{1em}

\begin{question} % Epperson Ej 1 pg 94
Realice tres iteraciones a mano del método de bisección aplicado a $f(x) = x^3 - 2$ en el intervalo $a = 0$ y $b=2$.
\end{question}

\begin{question} % Epperson Ej 2 pg 94
Para cada una de las funciones siguientes, encuentre la raíz con una precisión de 0.1 usando una calculadora (debería tomar a lo sumo cinco iteraciones):
\begin{enumerate}[a)]
    \item $f(x) = x - e^{-x²}$, $[a, b] = [0, 1]$.
    \item $f(x) = \ln x + x$,  $[a, b] = [1/10, 1]$.
    \item $f(x) = x^3 - 3$,  $[a, b] = [0, 3]$.
\end{enumerate}
\end{question}

\begin{question} % Epperson Ej 4 pg 95
Escriba un programa para resolver la ecuación $x = \cos x$. Elija el intervalo explorando el problema con una calculadora.
\end{question}

\begin{question} % Epperson Ej 5 pg 95
    Escriba un programa para resolver la ecuación $x = e^{-x}$. Elija el intervalo explorando el problema con una calculadora.
\end{question}

\begin{question} %Burden - Faires, Análisis Numérico. Ilustración pg 59
La ecuación $x^3 + 4 x^2 - 10 = 0$ tiene una raíz única en $[1, 2]$. Existen muchas formas de cambiar la ecuación para la forma de punto fijo $x = g(x)$ mediante una simple manipulación algebraica. Iterar en las siguientes representaciones de $g$ y verifique que el (eventual) punto fijo obtenido es una raíz de la ecuación inicial.
\begin{enumerate}[a)]
    \item $x = g_1(x) = x - x^3 - 4 x^2 + 10$
    \item $x = g_2(x) = \left( \frac{10}{x} - 4 x \right)^{1/2}$
    \item $x = g_3(x) = \frac{1}{2} (10 - x^3)^{1/2}$
    \item $x = g_4(x) = \left( \frac{10}{4 + x} \right)^{1/2}$
    \item $x = g_5(x) = x - \frac{x^3 + 4 x^2 - 10}{3 x^2 + 8 x}$
\end{enumerate}
\end{question}

\begin{question} % Burden - Faires, Análisis Numérico, p.8 pag 65
    Use el teorema de existencia y unicidad de punto fijo para mostrar que $g(x) = 2^{-x}$ tiene un único punto fijo en $[1/3, 1]$. Use una iteración de punto fijo para hallar el punto fijo con una precisión de $10^{-4}$. 
\end{question}

\begin{question} % Burden - Faires, Análisis Numérico, p.14 pag 66
    Use una iteración de punto fijo para determinar la solución de $x = \tan x$ con una precisión de $10^{-4}$, para $x \in [4, 5]$.
\end{question}

\begin{question} % Burden - Faires, Análisis Numérico, p.5 pag 75
    Use el método de Newton-Raphson para encontrar las soluciones con precisión $10^{-4}$ de las siguientes funciones:
\begin{enumerate}[a)]
    \item $x^3 - 2 x^2 - 5 = 0, \quad [1, 4]$
    \item $x - \cos x = 0, \quad [0, \pi/2]$
\end{enumerate}
\end{question}

\begin{question} % Burden - Faires, Análisis Numérico, p.17 pag 76
El polinomio de cuarto grado
\[ f(x) = 230 x^4 + 18 x^3 + 9 x^2 - 221 x - 9 \]
tiene dos ceros reales, uno en $[-1, 0]$ y otro en $[0, 1]$. Intente aproximar esos ceros con una precisión de $10^{-6}$ usando:
\begin{enumerate}[a)]
    \item bisección
    \item Newton-Raphson
\end{enumerate}
Use los intervalos como aproximaciones iniciales en a) y los puntos medios de esos intervalos en b). Compare la cantidad necesaria de iteraciones para cada caso.
\end{question}

\begin{question} % Burden - Faires, Análisis Numérico, p.25 pag 77
    La suma de dos números es $20$. Si a cada número se le suma su raíz cuadrada, el producto de los dos sumas es $155.55$. Determine los dos números con una precisión de $10^{-4}$.
\end{question}
\end{document}
